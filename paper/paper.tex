% Paper template for TAR 2022
% (C) 2014 Jan Šnajder, Goran Glavaš, Domagoj Alagić, Mladen Karan
% TakeLab, FER

\documentclass[10pt, a4paper]{article}

\usepackage{tar2023}

\usepackage[utf8]{inputenc}
\usepackage[pdftex]{graphicx}
\usepackage{booktabs}
\usepackage{amsmath}
\usepackage{amssymb}

\title{Stress Analysis in Social Media}

\name{Sven Šćekić, Marko Kuzmić, Lovro Bučar}

\address{
    University of Zagreb, Faculty of Electrical Engineering and Computing\\
    Unska 3, 10000 Zagreb, Croatia\\
    \texttt{\{sven.scekic, marko.kuzmic, lovro.bucar\}@fer.hr}\\
}


\abstract{
    Here we can write an abstract of the paper.
    The abstract is a paragraph of text ranging between 70 and 150 words.
}

\begin{document}

\maketitleabstract

\section{Introduction}

This section is the introduction to your paper.

\section{Related Work}

In scientific papers, this section usually (but not necessarily) briefly describes the related research and what makes the presented approach different from it.
Demonstration of citing~\citep{maguire-76}.

\section{Dataset}

A section that will be used to describe the dataset that was used in the research paper.

\section{Baseline Models}

This section which will describe the models that were used to determine the

\section{Results}

Finally, a section which describes the acquired results.

\section{Conclusion}

Conclusion is the last enumerated section of the paper.
It should not exceed half of a column and is typically split into 2--3 paragraphs.
No new information should be presented in the conclusion; this section only summarizes and concludes the paper.

\section*{Acknowledgements}

Here we can write a thank you to the professors and the assistants which were involved in TAR class.

\bibliographystyle{tar2023}
\bibliography{tar2023} 

\end{document}

