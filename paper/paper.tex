% Paper template for TAR 2022
% (C) 2014 Jan Šnajder, Goran Glavaš, Domagoj Alagić, Mladen Karan
% TakeLab, FER

\documentclass[10pt, a4paper]{article}

\usepackage{tar2023}

\usepackage[utf8]{inputenc}
\usepackage[pdftex]{graphicx}
\usepackage{booktabs}
\usepackage{amsmath}
\usepackage{amssymb}
\usepackage{multirow}

\title{Stress Analysis in Social Media}

\name{Sven Šćekić, Marko Kuzmić, Lovro Bučar}

\address{
    University of Zagreb, Faculty of Electrical Engineering and Computing\\
    Unska 3, 10000 Zagreb, Croatia\\
    \texttt{\{sven.scekic, marko.kuzmic, lovro.bucar\}@fer.hr}\\
}


\abstract{
    Here we can write an abstract of the paper.
    The abstract is a paragraph of text ranging between 70 and 150 words.
}

\begin{document}

\maketitleabstract

\section{Introduction}

This section is the introduction to your paper.

\section{Related Work}

In scientific papers, this section usually (but not necessarily) briefly describes the related research and what makes the presented approach different from it.

\section{Dataset}

The dataset that is used in this paper was introduced in a paper which collected posts from a social media webside called Reddit~\citep{dreaddit}.
On that website users can create posts in communities which focus on specific topics called subreddits.

The dataset consists of posts from subreddits where stressful topics are likely to be discussed in between January 1, 2017 and November 19, 2018.
These subreddits fall into one of these categories:

\begin{itemize}
    \item \textbf{Social}: The posts from this category are from the subreddit called \verb|r/relationships|.
        Posts from this subreddit talk about problems in a relationships, romantic or non-romantic.
    \item \textbf{Abuse}: posts from this category describe topics related to abuse in relationships and users share their experiences and advices.
        Subreddits that fall into this category are \verb|r/domesticviolence| and \verb|r/survivorofabuse|.
    \item \textbf{Anxiety}: subreddits from this category are \verb|r/anxiety| and \verb|r/stress|.
        Users in posts from this category talk about these mental illnesses, their symptoms, share advices and stories regarding these illnesses.
    \item \textbf{PTSD}: Same as the Anxiety category, posts from this category talk about mental illness, but focus on Post-Traumatic Stress Disorder.
        The only subreddit from this category is \verb|r/ptsd| and same as with posts from the Anxiety category, users share advices and stories and ask questions about the illness.
    \item \textbf{Financial}: posts from subreddits that fall into this category talk about difficult financial situations, share stories about homelessness and other stressful financial topics.
        Subreddits from this category are \verb|r/almosthomeless|, \verb|r/assistance|, \verb|r/food_pantry| and \verb|r/homeless|.
\end{itemize}

In the dataset there are in total 187,444 posts from these 10 subreddits.
The distribution of these posts can be seen in the Table~\ref{tab:dataset-distribution}.

\begin{table*}
    \caption{Distribution of collected posts}
    \label{tab:dataset-distribution}
    \begin{center}
        \begin{tabular}{|l|l|r|r|r|}
            \hline
            \textbf{Topic} & \textbf{Subreddit Name} & \textbf{Total Posts} & \textbf{Avg Tokens/Post} & \textbf{Labeled Segments}\\
            \hline
            Social                     & r/relationships    & 107,908 & 578 & 694 \\ \hline
            \multirow{2}{*}{Abuse}     & r/domesticviolence & 1,529   & 365 & 388 \\
                                       & r/survivorsofabuse & 1,372   & 444 & 315 \\ \hline
            \multirow{2}{*}{Anxiety}   & r/anxiety          & 58,130  & 193 & 650 \\
                                       & r/stress           & 1,078   & 107 & 78  \\ \hline
            PTSD                       & r/ptsd             & 4,910   & 265 & 711 \\ \hline
            \multirow{4}{*}{Financial} & r/almosthomeless   & 547     & 261 & 99  \\
                                       & r/assistance       & 9,243   & 209 & 355 \\
                                       & r/food\_pantry     & 343     & 187 & 43  \\
                                       & r/homeless         & 2,384   & 143 & 220 \\ \hline
        \end{tabular}
    \end{center}
\end{table*}

\subsection{Data Annotation}

A portion of the data was annotated using Amazon Mechanical Turk, a crowdsourcing marketplace that allows individuals to outsource jobs.

Primary job for annotators was to determine if there was stress present in the sentence that they were presented.
The definition for stress was taken from the Oxford English Dictionary states that stress is `a state of mental or emotional strain or tension resulting from adverse or demanding circumstance'.
Each annotator first had to take a qualification test to ensure that they would label the sentences correctly.
In the qualification test annotators were given instructions on how to correctly label post segments.

After that, each annotator was given 5 segments which they had to annotate with one of the following labels: ``Stress'', ``Not Stress'' or ``Can't Tell''.
In addition to the qualification test, each annotator was given one of the 50 `check questions' which were labeled by creators of the dataset.
If they did not label these `check questions' correctly their annotations were not included in the final dataset.

Since posts can often be longer, which would prove to be complicated for the annotators to label, they were divided into five-sentence chunks.
Added bonus of this technique is that the data can be used to determine the specific location of stress in the post.

In total, 3,553 labeled data points were collected out of which 39\% had perfect agreement.
With 52.3\% of the data being labeled as stressful, the dataset is nearly perfectly balanced.
Labeled data was split into two subsets: train and test subset.
The train subset contains 2,838 data points and the test subset consists of 715 data points.

\section{Baseline Models}

This section which will describe the models that were used to determine the

\section{Results}

Finally, a section which describes the acquired results.

\section{Conclusion}

Conclusion is the last enumerated section of the paper.
It should not exceed half of a column and is typically split into 2--3 paragraphs.
No new information should be presented in the conclusion; this section only summarizes and concludes the paper.

\section*{Acknowledgements}

Here we can write a thank you to the professors and the assistants which were involved in TAR class.

\bibliographystyle{tar2023}
\bibliography{tar2023} 

\end{document}

